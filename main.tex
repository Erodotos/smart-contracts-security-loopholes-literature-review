%
%                       This is a basic LaTeX Template
%                       for the Informatics Research Review

\documentclass[a4paper,11pt]{article}
% Add local fullpage and head macros
\usepackage{head,fullpage}     
% Add graphicx package with pdf flag (must use pdflatex)
\usepackage[pdftex]{graphicx}  
% Better support for URLs
\usepackage{url}
% Date formating
\usepackage{datetime}

\newdateformat{monthyeardate}{%
  \monthname[\THEMONTH] \THEYEAR}

\parindent=0pt          %  Switch off indent of paragraphs 
\parskip=5pt            %  Put 5pt between each paragraph  
\Urlmuskip=0mu plus 1mu %  Better line breaks for URLs


%                       This section generates a title page
%                       Edit only the following three lines
%                       providing your exam number, 
%                       the general field of study you are considering
%                       for your review, and name of IRR tutor

\newcommand{\examnumber}{2187344}
\newcommand{\field}{Detecting Ethereum Smart Contract Security Loopholes}
\newcommand{\supervisor}{Lorenzo Martinico}

\begin{document}
\begin{minipage}[b]{110mm}
        {\Huge\bf School of Informatics
        \vspace*{17mm}}
\end{minipage}
\hfill
\begin{minipage}[t]{40mm}               
        \makebox[40mm]{
        \includegraphics[width=40mm]{crest.png}}
\end{minipage}
\par\noindent
    % Centre Title, and name
\vspace*{2cm}
\begin{center}
        \Large\bf Research Methods In Security, Privacy, and Trust \\
        \Large\bf \field
\end{center}
\vspace*{1.5cm}
\begin{center}
        \bf \examnumber\\
        \monthyeardate\today
\end{center}
\vspace*{5mm}

%
%                       Insert your abstract HERE
%                       
\begin{abstract}

\end{abstract}

\vspace*{1cm}

\vspace*{3cm}
Date: \today

\vfill
{\bf Supervisor:} \supervisor
\newpage

%                                               Through page and setup 
%                                               fancy headings
\setcounter{page}{1}                            % Set page number to 1
\footruleheight{1pt}
\headruleheight{1pt}
\lfoot{\small School of Informatics}
\lhead{Informatics Research Review}
\rhead{- \thepage}
\cfoot{}
\rfoot{Date: \date{\today}}
%

\section{Introduction}

Ethereum is a general-purpose Blockchain, providing a platform to run
decentralized applications executing code called Smart Contracts. Smart Contracts
mainly manage valuable digital assets, and thus securing them is a top priority.
Yet, it is typical for any piece of code to have bugs. However, Smart Contract
bug fixes on the fly are not feasible since blockchain is an immutable
append-only data structure. Hence, detecting code bugs and vulnerabilities
before deploying Smart Contracts is vital.

In recent years there have been several attempts to create practical
vulnerability detection tools for Smart Contracts. This is a niche topic, and
there are several schools of thought when it comes to detecting security
loopholes. This literature review aims to probe associated studies, focusing on
different methods of detecting Smart Contract vulnerabilities, comparing
different approaches, and taxonomizing existing frameworks.

...

In this literature review, I explore the most prominent research attempts
towards creating an effective vulnerability detection tool for Ethereum Smart
Contracts. The method used to filter relevant studies comprised multiple steps.
Initially, I used trustworthy academic search engines such as Google Scholar and
IEEE Explorer to retrieve a few papers and have them in a paper pool. The main
selection criteria were the number of citations in combination with the paper
release year. Highly cited articles with recent release dates usually include
important research outcomes and can be the cornerstone for future work. After
isolating a few reputable studies, I used a graph representation tool
\cite{connectedpapers} that links relevant papers. This tool allowed me to
identify remarkable research papers rapidly. Afterward, I manually inspected the
search results and included the most reliable in my paper pool. Iterating the
procedure mentioned above, I converged into a set of papers to form this
literature review.

...

\section{Literature Review}

\subsection{White/Grey-box Fuzzing}

\subsubsection{Static Smart Contract Analysis}

Smart Check \cite{tikhomirov2018smartcheck} \\
Slither \cite{feist2019slither} \\
MadMax \cite{grech2018madmax} \\
Zeus \cite{kalra2018zeus} \\ 

\subsubsection{Dynamic Smart Contract Analysis}

Oyente \cite{luu2016making} \\
Manticore \cite{mossberg2019manticore} \\

\subsection{Black-box Fuzzing}

Contract Fuzzer \cite{jiang2018contractfuzzer} \\
ReGaurd \cite{liu2018reguard} \\

\subsection{Analysing Smart Contracts Using Formal Verification}

Securify \cite{tsankov2018securify} \\

\subsection{Analysing Smart Contracts Using Machine Learning}

SoliAudit \cite{liao2019soliaudit} \\

\section{Summary \& Conclusion}

\nocite{*}

%                Now build the reference list
\bibliographystyle{unsrt}   % The reference style
%                This is plain and unsorted, so in the order
%                they appear in the document.


\small

\bibliography{main}       % bib file(s).

\end{document}

